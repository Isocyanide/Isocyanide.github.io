\documentclass{article}
\usepackage{fancyhdr, amsmath, amssymb}
\pagestyle{fancy}
\lhead{MATH 9054A:\\ Exercise Sheet 5}
\chead{Udit Narayan}
\rhead{\today}

\begin{document}
\setlength{\parindent}{0cm}  
For positive integers $a.b$ where $gcd(a,b) = 1$, the equation $ax + by = n$ always has a solution in positive $x,y$ for all $n \geq ab$ \\

\textbf{Proof} \\

Since $gcd(a,b) = 1$, by the theory of linear Diophantine equations, we know that there always exists solutions (infinitely many of them) for $ax + by = n$. However we are only interested in the ones where both $x$ and $y$ are positive.\\

$ax + by = n$ \\
Let $n = ak + r$ where $r<a$ \\
$ax + by = ak + r$ \\
Set $x = k - t$ where $t$ takes integral values from $1$ to $b$ \\
$\implies a(k - t) + by = ak + r$ \\
$\implies - at + by = r$ \\
Let $r = 0$ \\
$\implies by = at$ \\
$\implies b|at$ \\
$\implies b|t$ (since $gcd(a,b) = 1$) \\
$\therefore t_{max} = b$ \\

Now lets consider the arbitrary case. We must prove that $-at + by = r$ has a solution $t$ in $[1,b]$
Here we employ the theory of Diophantine equations. \\
All the solutions to the above equation are $(t_0 + (b)d, y_0 + (a)d)$ where $(t_0, y_0)$ is an arbitrary solution of the same equation and $d$ is any integer. \\
We notice that the sequence of $t$'s follow an arithmetic progression with common difference b. Hence there always exists a $t$ between $1$ and $b$.\\
The corresponding $y$ is also positive as $by = at + r$.\\ 

However in order for $x$ to be positive, the minimum value of $k-t$ must be positive. \\
$\implies k-b\geq 0$\\
$\implies k\geq b$\\

Since $n = ak + r$.\\
For $n \geq ab$ we always have a positive solution in $x,y$. \\

Remark : This has been proved for all $n > ab-a-b$ but this proof is rather elementary compared to that and this result is still quite powerful.    

\end{document}
